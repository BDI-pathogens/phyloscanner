\documentclass{article}
\usepackage{graphicx}
\usepackage[a4paper, total={16cm, 24cm}]{geometry}
\usepackage{xspace}
\usepackage[hidelinks]{hyperref}
\usepackage{color}
\newcommand{\p}{\texttt{phyloscanner}\xspace}
\newcommand{\pmt}{\texttt{phyloscanner\_make\_trees.py}\xspace}
\newcommand{\pat}{\texttt{phyloscanner\_analyse\_trees.R}\xspace}
\newcommand{\pcl}{\texttt{phyloscanner\_clean\_alignment.R}\xspace}
\let\c\texttt
\newcommand{\www}{\color{blue} \underline}
\newcommand{\R}{\c{RAxML}\xspace}
\newcommand{\iq}{\c{IQtree}\xspace}

\title{\p\\available from \href{https://github.com/BDI-pathogens/phyloscanner}{\www{github.com/BDI-pathogens/phyloscanner}}}
\date{This manual last updated \today}
\author{Coding lead by Chris Wymant and Matthew Hall,\\with contributions from Oliver Ratmann and Christophe Fraser}

\begin{document}
\maketitle


\begin{figure}[!h]
\centering
\includegraphics[width=0.95\textwidth]{PhyloscannerDiagram_big4.pdf}
\end{figure}

\vspace*{5mm}

\p analyses pathogen genetic diversity and relationships between and within hosts at once, in windows along the genome using mapped {\it reads} (fragments of DNA produced by next-generation sequencing), or using phylogenies with multiple sequences per host previously generated by the user via any method.
An example showing its action on simulated data (which is included with the code) can be seen at the GitHub repository linked to above.

\vspace*{5mm}

If you have a query about \p, ask it publicly at the \href{https://groups.google.com/forum/\#!forum/phyloscanner-users}{\www{google group}}.  
If you have a problem getting the code to run or think you've found a bug, please create a \href{https://github.com/BDI-pathogens/phyloscanner/issues}{\www{\it New issue}} on the GitHub repository.
The usual bug-reporting etiquette applies.
For non-trivial bugs, to allow us to reproduce the problem it's helpful to have input that causes the problem.
This needn't be all your data.
For example if one read in a bam file is causing a problem, you could make a new bam file containing only that read, allowing us to see the problem very quickly.
Isolate the problem as much as you can first -- help us to help you!
If you get an error when you run \p as part of a larger piece of code you've written yourself (e.g. a job submission script to a cluster), please don't make us debug that -- the error might be in your code rather than ours.
Isolate the problem to the relevant \p command.

\vspace*{5mm}

If you use \p, please cite it~\cite{Wymant157768} and its dependencies: \c{SAMtools}~\cite{Li08062009}, \c{MAFFT}~\cite{Katoh15072002}, \R~\cite{doi:10.1093/bioinformatics/btu033} and \c{ggtree}~\cite{MEE3:MEE312628}.
Details of these are found in the file \c{CitationDetails.bib} in \p's \c{InfoAndInputs} subdirectory.

\vspace*{5mm}

Throughout the manual we'll use {\it phylogeny} and {\it tree} interchangeably.
Forgive us.

\newpage
\tableofcontents
\newpage

\section*{A tool in two parts}
Conceptually, \p can be neatly split into two steps.
Firstly, inferring phylogenies that contain both within- and between-host diversity, in windows along the genome, using mapped reads as input.
Secondly, analysing those phylogenies or other similar ones (i.e. containing within- and between-host diversity, but created however the user likes) in order to infer interesting things like transmission and multiply infected individuals, and give summaries of diversity.

This conceptual split is reflected in the code: \pmt runs the first step and \pat runs the second step.
Why didn't we merge the two steps into one?
\begin{itemize}
\item For some uses, step one may not be needed.
For example if you don't have mapped reads, but have sequence data in another format.
We presented an example of this in the \p paper: a dataset consisting of multiple whole genomes of {\it Streptococcus pneumoniae} for each person carrying the bacterium, obtained by separately sequencing multiple colonies.
\item For some uses, step two may not be needed.
Having extracted the within- and between-host diversity information in each window (first as a multiple sequence alignment and then as a phylogeny), you might want to do your own totally different analysis of it.
%Step one consists of extracting the right reads from all samples for each window, processing them, aligning them, then running \R on the aligned reads.
\item Each step is controlled by parameters and options to tailor its usage to different kinds of data, and we don't (yet) have a way to automatically optimise these for each data set, so you will need to have a little play to explore what's appropriate in your case.
If you are running both steps, it's best to think first about getting step one right first, and then getting step two right.
Step one is primarily about bioinformatics, step two about phylogenetic analysis.
\end{itemize}

Note that to run either \pmt or \pat, like running any executable file, you must either specify the full path to the executable (e.g.\\\c{/path/to/my/phyloscanner/code/}\pmt) or add the directory that contains the executable to the \c{\$PATH} environment variable in your terminal (google this if needed).
In the manual we'll refer to scripts named like \c{tools/SomeCode}; such scripts live in the \c{tools} subdirectory inside the main \p code directory.


Fun fact: step one was written by Chris, step two mostly by Matthew.
We sat next to each other while coding, and we hope that's reflected in how easy it is to make the two steps work together.
If it isn't, let us know.
%While the two steps can be run separately, we had their combined use  


\part{Inferring within- and between-host phylogenies in windows along the genome, using mapped reads (Chris)}

\section{The basic command}

As input to \p, you need to have generated files of mapped reads in bam format -- bam files henceforth.
The bam format does not include the sequence to which the reads were mapped -- the {\it reference} -- which we also need.
With the initial \c{\$} traditionally indicating that what follows is a command to be run on the command line / in a terminal, the basic \p command looks like

\begin{verbatim}
$ phyloscanner.py ListOfMyInputFiles.csv --windows 1,300,301,600,...
\end{verbatim}
where
\begin{itemize}
\item \c{ListOfMyInputFiles.csv} is a plain-text, comma-separated-variable (csv) format file in which the first column contains the bam files, the second column contains the corresponding reference files.
An optional third column, if present, contains {\it aliases} -- things to rename each bam file to in \p output; if not present the base name of the bam, i.e. the file name not including the directory path, will be used.
e.g. This input file list might look like\\
\c{PatientA.bam,PatientA\_ref.fasta,A\\
PatientB.bam,PatientB\_ref.fasta,B}\\
Bam file base names, and aliases if present, must be unique and free of whitespace.
Quotation marks are interpreted as wrapping/protecting fields, allowing them to contain commas that are not field separators.
You can generate this input file list however you like, including manually.
The following block of code illustrates how you might generate it automatically from the command line.
It also checks that your files exist, which \p does anyway, but errors are best caught soonest.
\begin{verbatim}
$ for ID in PatientA PatientB PatientC; do
$   bam=MyBamsDir/"$ID".bam
$   ref=MyRefsDir/"$ID".fasta
$   if [[ ! -f "$bam" ]]; then
$     echo "$bam does not exist" >&2
$     break
$   elif [[ ! -f "$ref" ]]; then
$     echo "$ref does not exist" >&2
$     break
$   fi
$   echo "$bam","$ref","$ID"
$ done > MyPhyloscannerInputFile.csv
\end{verbatim}
If those IDs were stored in a plain-text file \c{MyIDs.txt} separated by whitespace, you could replace the first line of that loop with\\
\c{for ID in \$(cat MyIDs.txt); do}
\item The \c{--windows} (or \c{-W}) option is used to specify an even number of comma-separated positive integers: these are the coordinates of the windows to analyse, interpreted pairwise, i.e. the first two are the left and right edges of the first window, the third and fourth are the left and right edges of the second window, $\ldots$ i.e. in the above example we have windows 1-300, 301-600, $\ldots$
\end{itemize}
Note that to run the \c{phyloscanner.py} binary file, like running any other binary file, you must either prepend to it the directory in which it lives (e.g. \c{/path/to/my/phyloscanner/code/phyloscanner.py}) or add that directory to the \c{\$PATH} environment variable associated with your terminal (google this if needed). 

\section{What do the window coordinates mean exactly?}
By default the references used for mapping, together with any extra references if included using the \c{--alignment-of-other-refs} option, are all aligned together and window coordinates are interpreted with respect to the alignment (i.e. position $n$ refers to the $n$th column of that alignment, which could be a gap for some of the sequences).
This alignment can be found in the file \c{RefsAln.fasta} after running \p, should you want to inspect it and possibly run again.
You can manually specify window coordinates with respect to this alignment, using the \c{--windows} option, or have windows automatically chosen using \c{--auto-window-params}, which attempts to minimise the affect of insertions and deletions in the references on your window width and overlap preferences.
Alternatively, if you are using the \c{--alignment-of-other-refs} option to include extra references, you can use \c{--pairwise-align-to} to name one of these references to be a kind of {\it reference reference}: instead of aligning of all the bam file references to each other, they will be sequentially and separately pairwise-aligned to your named reference, and window coordinates are interpreted with respect to that named reference.
Using the \c{--pairwise-align-to} option is expected to more stable than \c{--windows} or \c{--auto-window-params} if your bam file references are many and diverse, since pairwise alignment is easier than multiple sequence alignment.
It also has the advantage that when running \p more than once with different bam files, the coordinates mean the same thing each time.

\section{What windows should I choose?}
I'm glad you asked.
It's important.

\subsection{Start \& end}

You might as well fully cover the genomic region you're interested in.
That requires choosing where to start and where to end.
If you're interested in the whole genome, the start is 1 and the end is the genome length, or more precisely the length of the alignment of all references together (or the length of your named reference if you used the \c{--pairwise-align-to} option.)
You may know that your reads don't start right at the beginning of the genome.
If this is the case, a good place to start your first window would be the genome position at which you start having reads.

If your reads were generated by amplifying the sample using primers, the primers ought to have been trimmed from the reads as part of whatever bioinformatic pipeline produced your input bam files.
(This can be done for example using \href{https://github.com/sanger-pathogens/Fastaq}{\color{blue} \underline{\c{fastaq}}}, which is called as part of the \href{https://github.com/ChrisHIV/shiver}{\c{\color{blue} \underline{shiver}}} pipeline.)
Then a sensible choice for the start of the first window would be the first position after the first primer, and a sensible choice for the end of the last window would be the last position before the last primer.

{\it You might as well fully cover the genomic region you're interested in.}
Let's revisit that.
In what part of the genome would you like to do phylogenetics?
Perhaps all of it.
Then again, remember that phylogenetics assumes neutral evolution at every included site.
Though \p allows individual specified sites to be excised after alignment of all the reads from all the bam files, there may be regions of the genome where selected sites are so dense that it's simpler to skip the region altogether.
There may also be regions of the genome where there are so many insertions and deletions (indels) that you're skeptical that the reads will align correctly in the first place.
You can choose to specify no windows covering such regions. 

\subsection{Window width}
As well as choosing where your first window should start and where your last one should end, you need to choose how wide each single window is.
If a window is very small, so little diversity is contained inside it (within or between samples) that the number of {\it unique} reads overlapping the window is small, hindering meaningful phylogenetics.
If window width exceeds read length, then you will have no reads in the window, since we keep only reads fully overlapping the window.
Somewhere between these two extremes therefore maximises the number of unique reads; to help you figure out what that is for your data, you can run \p with the \c{--explore-window-widths} option.
This reports, for each in a list of window widths to try, how many unique reads are found for each bam and at each position along the genome.
To summarise these read counts into a single value that varies with window width, you could use the mean or the median, or you might be interested in a percentile lower than the 50th, if your concern is ensuring some minimal amount of diversity across all bams and all genomic positions.
Up to you.

(Note that some of the other options can affect how many reads you get in a window and so can affect what \c{--explore-window-widths} will tell you, namely the \c{--excision-coords}, \c{--merging-threshold}, \c{--min-read-count}, \c{--quality-trim-ends}, \c{--min-internal-quality}, and \c{--discard-improper-pairs} options.
The first two can result in two or more unique reads being merged into one; the rest can simply discard some reads.
You could choose values for the associated parameters immediately and then use \c{--explore-window-widths}, or else come back to \c{--explore-window-widths} later on once you've got the hang of \p and investigated the effect of those other options in your data.)  

Power users might want to optimise their own measure of phylogenetic information as a function of window width; one of the first metrics to pop into your head might be the mean bootstrap of all nodes in the tree.
That's not advised because within a sample there may be many very similar sequences, and the set of nodes connecting these may have poor boostrap support, but this is not something that ought to be penalised.
Also in theory you might be able to increase the window width until only a single read is found spanning the window in each patient; your bootstraps might then be great, because between-host diversity is greater than that within-host, but you've thrown out all the within-host information.

NB wherever {\it read} and {\it read length} appeared in the discussion above, they should be substituted for {\it insert} and {\it insert size} if you have paired-read data AND the reads in a pair sometimes overlap AND you run \p with \c{--merge-paired-reads} to merge overlapping paired reads into a single longer read (see the cartoon below).
A complication with this is that whereas read length is typically fixed within a sample, insert size has a distribution of different values.
A window which is wider than twice the read length can never get any reads, because the reads in a pair need to overlap in order to be merged.
So you have two choices.
\begin{enumerate}
\item Choose\\  
(read length) $<$ (window width) $<$ (twice the read length)\\  
Then you're restricted to the subset of read pairs that satisfy  
(window width) $\leq$ (insert size) $<$ (twice the read length)  
because only such pairs can overlap and fully span the window.
The fraction of such reads in a sample is the integral of the unit-normalised insert size distribution between the two limits in the inequality above.
\item Choose\\  
(window width) $\leq$ (read length)\\  
Then you can have single reads contribute in addition to merged overlapping read pairs.
But perhaps that window is too short; see the discussion on window width above.
\end{enumerate}

\subsection{Window overlap}
How much should neighbouring windows overlap?
A simple answer to this is zero, i.e. each window starts right after the previous one ends.
e.g. 1-99, 100-199, 200-299, $\ldots$\\
{\bf TODO: discuss positive overlap.}

\section{Optional arguments}

Information about all optional arguments and what they do can also be seen by running \p with the \c{--help} option.
That will give information guaranteed to be synchronised to your current version of the code, as well as showing all option shorthands (e.g. \c{--help} = \c{-h}); however we go into slightly more detail here.

\subsection{Window options}
You must choose exactly one of these: \c{--windows}, \c{--auto-window-params}, \c{--explore-window-widths}.
\begin{itemize}
\item \c{--windows}: a comma-separated series of paired coordinates defining the boundaries of the windows.
e.g. specifying \c{1,300,301,600,601,900} would define windows 1-300, 301-600, 601-900.
\item \c{--auto-window-params}: used to specify 2, 3 or 4 comma-separated integers controlling the automatic creation of regular windows.
The first integer is the width you want windows to be, weighting each column in the alignment of bam file references (plus any extra references included with \c{--alignment-of-other-refs}) by its non-gap fraction, so that windows with rows or columns containing many gaps become correspondindly wider.
The second is the overlap between the end of one window and the start of the next (which can be negative, implying unused space in between windows; the recommended value of 0 means each window starts right after the previous one ends).
The optional third integer is the start position for the first window (by default, 1).
The optional fourth integer is the end position for the last window (by default, windows will continue up to the end of the alignment of references).
This option cannot be used with \c{--pairwise-align-to}.
\item \c{--explore-window-widths}: use this option to explore how the number of unique reads found in each bam file in each window, all along the genome, depends on the window width.
After this option specify a comma-separated list of integers.
The first integer is the starting position for stepping along the genome, in case you're not interested in the very beginning.
Subsequent integers are window widths to try.
For example, if you specified 1000,100,150,200 we would count the number of unique reads in windows 1000-1099, 1100-1199, 1200-1299, ... and in 1000-1149, 1150-1299, 1300-1449 ... and in 1000-1199, 1200-1399, 1400-1599, ... where the dots denote continuation to the end of the genome.
Output is written to the file specified with the \c{--explore-window-width-file} option.
\item \c{--explore-window-width-file}: used to specify an output file for window width data, when the \c{--explore-window-widths} option is used.
Output is in in csv format.
\end{itemize}

\subsection{Recommended options}
%Options we think are particularly worth your while understanding and using.
\begin{itemize}
\item \c{--alignment-of-other-refs}: used to specify an alignment of reference sequences for inclusion with the reads, for comparison.
These references need not be those used to produce the bam files.
This option is required if phyloscanner is to analyse the trees it produces.
\item \c{--pairwise-align-to}: by default, \p figures out where corresponding windows are in different bam files by creating a multiple sequence alignment containing all of the mapping references used to create the bam files (plus any extra references included with \c{--alignment-of-other-refs}), and window coordinates are intepreted with respect to this alignment.
However using this option, the mapping references used to create the bam files are each separately pairwise aligned to one of the extra references included with \c{--alignment-of-other-refs}, and window coordinates are interpreted with respect to this reference.
\item \c{--x-raxml}: use this option to tell phyloscanner how to run \c{RAxML}; by default, \c{'raxmlHPC-AVX -m GTRCAT -p 1'}.
You will need to change the first part if your \r binary is not called \c{raxmlHPC-AVX}, or if the binary's location is not contained in your \c{\$PATH} environment variable (i.e. if you need to specify the path to the binary in order to run it).
\c{-m} tells \R which evolutionary model to use, and \c{-p} specifies a random number seed for the parsimony inferences; both are compulsory.
You may include any other \R options in this command.
The set of things you specify with \c{--x-raxml} need to be surrounded with one pair of quotation marks (so that they're kept together as one option for \p and only split up for \R).
If you include a path to your \R binary, it may not include whitespace, since whitespace is interpreted as separating \R options.
Do not include options relating to bootstraps: use \p's \c{--num-bootstraps} and \c{--bootstrap-seed} options instead.
Do not include options relating to the naming of files.
\item \c{--check-recombination}: calculates a metric of recombination for each
sample's set of reads in each window.
(Recommended only if you're interested, of course.)
How the metric is calculated: for each possible set of three sequences, one is considered the putative recombinant and the other two the parents.
For each possible crossover point (the point at which recombination occurred), we calculate $d_L$ as the difference between the Hamming distance from the recombinant to one parent and the Hamming distance from the recombinant to the other parent, looking to the left of the crossover point only; similarly we calculate $d_R$ looking to the right of the crossover point only.
$d_L$ and $d_R$ are signed integers, such that their differing in sign indicates that the left and right sides of the recombinant look like different parents.
We maximise the difference between $d_L$ and $d_R$ (over all possible sets of three sequences and all possible crossover points), take the smaller of the two absolute values, and normalise it by half the length of the alignment of sequences.
This means that the maximum possible score of 1 is obtained if and only if the two parents disagree at every site, the crossover point is exactly in the middle, and either side of the crossover point the recombinant agrees perfectly with one of the parents e.g.\\
AAAAAAA\\
AAAACCC\\
CCCCCCC\\
Calculation time scales cubically with the number of unique sequences each sample has per window, and so the option is turned off by default.
You can save time by only turning it on only after you've settled on the values of other parameters that affect the number of unique sequences per window (notably window width, a merging threshold and a minimum
read count).
\end{itemize}

\subsection{Read quality options}
\begin{itemize}
\item \c{--discard-improper-pairs}: discard all reads that are were flagged, at the time of mapping, as improperly paired: in the wrong orientation, or one mate unmapped, or too far apart.
For paired-read data.
\item \c{--quality-trim-ends}: each end of the read is trimmed inwards until a base of this quality is met.
\item \c{--min-internal-quality}: discard reads containing more than one base of a quality below this
parameter.
(If used in conjuction with the \c{--quality-trim-ends option}, the trimming of the ends is done first.)
\item \c{--min-read-count}: reads with a count (i.e. the number of times that sequence was observed,
after merging if merging is being done) less than this value are discarded.
The default value of 1 means all reads are kept.
You might want to discard rare reads to protect against sequencing error.
Retaining fewer reads will also speed up all subsequent processing and analysis of the reads.
\end{itemize}

\subsection{Other assorted options}
\begin{itemize}
 \item 
\end{itemize}





\part{Analysing within- and between-host phylogenies (Matthew)}



\bibliographystyle{ieeetr}
\bibliography{CitationDetails}

\end{document}
